\chapter{HPC and Exascale}


\section{Introduction}

The HPC world is spread over all the scientific domains nowaday and is used from usual computation to the biggest black holes merging simulations. 
Firstly, we present the rules, laws and organization that characterize the HPC world from the Moore's law, Amdhal with the Flynn Taxonomy and the description of the recent bottlenecks and walls. 

Considering homogeneous cluster only. 

\section{Parallelism}



\subsection{Flynn taxonomy}

The flynn taxonomy presents a hierarchical organization of computation machine.

In this classification \cite{flynn1972some}, Michael J. Flynn present the SIMD,SISI, MISD and MIMD.
Add table and present some example of machines

\subsection{Goals}

\subsubsection{Speedup}

Speedup can be separate in two parts, Latency and Throughput.

\subsection{Bottlenecks}

\subsection{Amdahl and Gustafson}

The Amdahl's\cite{amdahl1967validity} law is use to find the theoretical speedup in latency of a program.
We can separate a program in two parts, the one that can be execute in parallel and the one that is sequential. 
And even if we reduce the parallel part to infinite the sequential part will reach 100\% of the total time. 

Extracted from the Amdahl paper the law can be writen as: 

\begin{equation}
Speedup = \frac{1}{Seq + \frac{Par}{n}}
\end{equation}

Where $Seq + Par = 1$ and $Seq$ and $Par$ respectively the sequential and parallel ratio of a program. 

On plot XXX, a representation of the perfect speedup is represented.
This law does not take every cases in acount. 
Indeed the speedup can be much better if we grow the amount of work to be done in the same time as the number of processes that execute the program. 

This is proposed by the Gustafson law presented in. 

\section{Hardware}

The structure of an HPC cluster can be 

\subsection{Classical CPU}

\subsection{GPU}

\subsection{FPGA and ASICS}

\section{Clusters and Exascale}

\subsection{Benchmarking}

\subsubsection{TOP500}

The TOP500 list is a list of the 500 most powerful super computers in the world.
The ranking is based on the LINPACK and LAPACK suite. 
Initialy the LINPACK, Linear Algebra library was used but is now replaced by the new Linear Algreba Pack, LAPACK. 

\subsection{Composition and usage}

\subsection{Interconnection}

\section{Languages}

\subsection{Accelerators}

\subsection{Runtimes}

\section{Optimization}


\subsubsection{Memory locality}

\subsubsection{Vectorization}

\subsection{CPU specifications}

\subsection{GPUs specifications}

\subsection{Communications}

\section{Conclusion}
