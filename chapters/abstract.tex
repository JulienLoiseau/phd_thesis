\hrule
\vspace{.2cm}
Titre français: \textbf{\phdTitleFR}
\vspace{.2cm}
\hrule
\vspace{.3cm}

La course à l'Exascale est entamée et tous les pays du monde rivalisent pour présenter un supercalculateur exaflopique à l'horizon 2020-2021. 
Ces superordinateurs vont servir à des fins militaires, pour montrer la puissance d'une nation, mais aussi pour des recherches sur le climat, la santé, l'automobile, physique, astrophysique et bien d'autres domaines d'application. 
Ces supercalculateurs de demain doivent respecter une enveloppe énergétique de 1 MW pour des raisons à la fois économiques et environnementales. 
Pour arriver à produire une telle machine, les architectures classiques doivent évoluer vers des machines hybrides équipées d'accélérateurs tels que les GPU, Xeon Phi, FPGA, etc. 

Nous montrons que les benchmarks actuels ne nous semblent pas suffisants pour cibler ces applications qui ont un comportement irrégulier. 
Cette étude met en place une métrique ciblant les aspects limitants des architectures de calcul: le calcul et les communications avec un comportement irrégulier. 

Le problème mettant en avant la complexité de calcul est le problème académique de Langford. 
Pour la communication nous proposons notre implémentation du benchmark du Graph500.
Ces deux métriques mettent clairement en avant l'avantage de l'utilisation d'accélérateurs, comme des GPUs, dans ces circonstances spécifiques et limitantes pour le HPC. 
Pour valider notre thèse nous proposons l'étude d'un problème réel mettant en jeu à la fois le calcul, les communications et une irrégularité extrême. 
En réalisant des simulations de physique et d'astrophysique nous montrons une nouvelle fois l'avantage de l'architecture hybride et sa scalabilité. 

\vspace{.3cm}
\hrule
\vspace{.1cm}

{
\small
Mots clés: Calcul Haute Performance, Architecture Hybrides, Simulation
}

\vspace{.1cm}
\hrule

\vspace{.4cm}
\hrule
\vspace{.2cm}
English title: \textbf{\phdTitleEN}
\vspace{.2cm}
\hrule
\vspace{.3cm}

The countries of the world are already competing for Exascale and the first exaflopics supercomputer should be release by 2020-2021.
These supercomputers will be used for military purposes, to show the power of a nation, but also for research on climate, health, physics, astrophysics and many other areas of application.
These supercomputers of tomorrow must respect an energy envelope of 1 MW for reasons both economic and environmental.
In order to create such a machine, conventional architectures must evolve to hybrid machines equipped with accelerators such as GPU, Xeon Phi, FPGA, etc.

We show that the current benchmarks do not seem sufficient to target these applications which have an irregular behavior.
This study sets up a metrics targeting the walls of computational architectures: computation and communication walls with irregular behavior.
The problem for the computational wall is the Langford's academic combinatorial problem.
We propose our implementation of the Graph500 benchmark in order to target the communication wall.

These two metrics clearly highlight the advantage of using accelerators, such as GPUs, in these specific and representative problems of HPC.
In order to validate our thesis we propose the study of a real problem bringing into play at the same time the computation, the communications and an extreme irregularity.
By performing simulations of physics and astrophysics we show once again the advantage of the hybrid architecture and its scalability.

\vspace{.3cm}
\hrule
\vspace{.1cm}

{
\small
Key works: High Performance Computing, Hybrid Architectures, Simulation
}

\vspace{.1cm}
\hrule

\vspace{.3cm}
\textbf{Discipline: \phdDiscipline}

\textbf{Spécialité: \phdSpeciality}
\vspace{.2cm}

\hspace{8cm}Université de Reims Champagne Ardenne

\hspace{8cm}Laboratoire du CReSTIC EA 3804

\hspace{8cm}UFR Sciences Exactes et Naturelles, 

\hspace{8cm}Moulin de la Housse, 51867 Reims

